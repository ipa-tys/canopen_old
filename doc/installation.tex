\chapter{Installation}
\label{chap:installation}

\section{CAN device driver}

Currently, the library has only been tested with the PCAN-USB CAN interface for USB from Peak System.It has been tested with version 7.5. The Linux user manual is available at: \url{http://www.peak-system.com/fileadmin/media/linux/files/PCAN%20Driver%20for%20Linux_eng_7.1.pdf}. Briefly, to install the drivers under Linux, proceed as follows:
  \begin{itemize}
  \item Download and unpack the driver: \url{http://www.peak-system.com/fileadmin/media/linux/files/peak-linux-driver-7.5.tar.gz}.
  \item \texttt{cd peak-linux-driver-x.y}
  \item \texttt{make clean}
  \item Use the chardev driver: \texttt{make NET=NO}
  \item \texttt{sudo make install}
  \item \texttt{/sbin/modprobe pcan}


  \item Test that the driver is working:
    \begin{itemize}
    \item \texttt{cat /proc/pcan} should look like this, especially \texttt{ndev} should be \texttt{NA}:
{\scriptsize
\begin{verbatim}
*------------- PEAK-System CAN interfaces (www.peak-system.com) -------------
*-------------------------- Release_20120319_n (7.5.0) ----------------------
*---------------- [mod] [isa] [pci] [dng] [par] [usb] [pcc] -----------------
*--------------------- 1 interfaces @ major 248 found -----------------------
*n -type- ndev --base-- irq --btr- --read-- --write- --irqs-- -errors- status
32    usb -NA- ffffffff 255 0x001c 0000cc3f 0000edd1 00063ce1 00000005 0x0014
\end{verbatim}}
\item \texttt{./receivetest -f=/dev/pcan32} Turning the CAN device power on and off should trigger some CAN messages which should be shown on screen.
    \end{itemize}
  \end{itemize}

  \section{ROS-indendent IPA CANopen library}

  \begin{itemize}
    \item \texttt{git clone \url{git://github.com/ipa-tys/canopen.git}}
    \item \texttt{cd driver}
    \item \texttt{make}
    \item \texttt{sudo make install}. Currently, this copies the archive \texttt{libcanopen.a} into \texttt{/user/lib}, the header files \texttt{canopen\_internal.h}, \texttt{canopenmsg.h}, \texttt{canopen\_highlevel.h}, \texttt{chain.h}, and \texttt{canopenmaster.h} into \texttt{/usr/include}. Files containing definitions of the CANopen standard (\texttt{indices.csv}, \texttt{constants.csv}, \texttt{PDOs.csv}) are copied into \texttt{~/.canopen}.
    \item Now, build the demos: \texttt{cd ../demo}
      \item \texttt{make}.
    \item Test if the installation was successful: \texttt{./single\_device}. This should give the output:
      {\scriptsize
\begin{verbatim} Please call the program with the CAN deviceID,e.g. './single\_device 12'\end{verbatim}}
\end{itemize}

\section{IPA CANopen ROS package}

Currently, this requires that you first perform the two installation steps (PCAN driver, ROS-independent CANopen library) above manually. Then:

\begin{itemize}
\item \texttt{git clone \url{git://github.com/ipa-tys/ros_canopen.git}}
\item \texttt{rosmake ros\_canopen}. Make sure dependencies are installed (e.g. package cob\_srvs from stack cob\_common).
\item Test if the installation was successful:
\begin{itemize}
\item In one terminal: \texttt{roscore}
\item In another terminal: \texttt{rosrun ros\_canopen ros\_canopenmasternode}. This should give the output:
 {\scriptsize
\begin{verbatim}File not found. Please provide a description file as command line argument\end{verbatim}}
\end{itemize}


\end{itemize}


