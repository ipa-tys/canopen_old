\chapter{Using the ROS-independent library}
\label{chap:rosindependentlibrary}

In this chapter, we show how to use the framework-independent {\em canopen} library. We assume that you have (1) installed the CAN device driver, and (2) built the {\em canopen} library, as described in Sections \ref{chap:installation:devicedriver} and \ref{chap:installation:canopen}, respectively.

\section{A ``Hello world'' example: homing a motor device}

Before going into details, let's start with a little example. This example also serves as a useful test of whether everything is ok with your installation and with your CAN device driver.

Go to the \texttt{build/examples} directory in your \texttt{canopen} folder. Here you can see a file called \texttt{homing}. This program does nothing else than its name implies, homing (referencing) a CANopen device.
If you call it without arguments, it will tell you it needs to know
\begin{itemize}
\item The device file of your CAN bus.
\item The CANopen ID of the device that you want to reference, in decimal (not hexadecimal!) notation.
\end{itemize}

Let's assume you CAN bus device file is \texttt{/dev/pcan32} and the CANopen ID of your device is \texttt{12}. Now call the program with:
\begin{verbatim}
./homing /dev/pcan32 12
\end{verbatim}

If you hear or see your device doing something, e.g. a clicking noise, or some small motions, then everything has worked out correctly and you have successfully used the {\em canopen} library! The device now assumes that its current position is $0$.

Before we look at the source code of this example, let's look at the different levels of abstraction in which the {\em canopen} library can be used.

\section{An overview of the library}

The library can be used on two different levels:
\begin{itemize}
\item By directly writing CANopen messages to the bus using the API functions provide in \texttt{canopen\_highlevel.h}.
\item On a higher level, there is also an implementation of a master thread with functionality to manage device objects and object groups.
\end{itemize}

We first describe the use of API functions to write directly to the CAN bus. This chapter assumes that you have completely all the installation steps described in Chapter \ref{installation}, including the building of the examples in the \texttt{demo} folder.

\section{Using API functions to directly write to the CAN bus}

The whole library lives within a dedicated namespace, \texttt{canopen}. Therefore, you will often see this as a prefix, e.g. \texttt{canopen::openConnection}.

\subsection{Getting started: Communicating with a single device (demo/single\_device.cpp)}

This example shows how to communicate with a single device. If you invoke the script with the CAN ID (in decimal representation) of your device as a command-line parameter, e.g. \texttt{./single\_device 12}, you should first see your item moving around until it is referenced (homing mode). Then the device will start moving in one direction in interpolated position (IP) mode. To understand how to use the library let's examine the code of the example\footnote{In this manual, only the relevant parts of the examples are shown. Go to the \texttt{demo} folder to see the full example code.}.

\begin{verbatim}
#include <canopen_highlevel.h>
\end{verbatim}
This is the header file to include if you just want to use API functions to directly write to the CAN bus, without using a master thread to manage device objects and object groups.

\begin{verbatim}
if (!canopen::openConnection("/dev/pcan32"))
\end{verbatim}
This is always the first command to invoke when using the CANopen library. It attempts to open the device connection. It returns \texttt{true} in case of success and \texttt{false} otherwise.

\begin{verbatim}
canopen::initListenerThread();
\end{verbatim}
Even in the direct communication without a {\em master thread}, there needs to be a separate thread to listen for incoming messages from the CAN bus. Without a master, this thread only performs the following action with the two types of messages coming in from the devices:
\begin{itemize}
\item {\em SDOs}: An SDOs coming in from a device is always a response to an SDO sent out from the master to a device. In this library, the user is shielded from this asynchronous communication. To the user, SDO communication simply appears as function calls with the return value of the function being the SDO reply coming from the device in response to the SDO sent out from the master. 
\item {\em PDOs}: PDOs coming in from the devices (tPDOs) are simply taken by the listener thread and put into the queue \texttt{canopen::incomingPDOs}. When the master thread is used, it takes care of distributing the PDOs to the device objects and of interpreting them. Without using a master, users are responsible by themselves to take the incoming PDOs out of the queue and to interpret (or discard) them. Without any user action, the \texttt{canopen::incomingPDO} queue will grow indefinitely.
\end{itemize}

\begin{verbatim}
canopen::initNMT();
\end{verbatim}
This puts the communication (301 standard) state machines of all devices on the bus into operational mode.

\begin{verbatim}
canopen::initNMT();
\end{verbatim}

\begin{verbatim}
if (!canopen::initDevice(deviceID))
\end{verbatim}
This puts the motor (402 standard) state machine of a device into operational mode and return true if the device reports itself as operational afterwards, and false otherwise.

\begin{verbatim}
if (!canopen::homing(deviceID))
\end{verbatim}
This triggers a device to reference itself (homing). The function call blocks until the drive has stopped moving and then returns true if the device reports itself as referenced and false otherwise.

\begin{verbatim}
if (!canopen::enableIPmode(deviceID))
\end{verbatim}
This makes a device ready to perform motion in the interpolated-position (IP) mode by receiving position commands via PDOs. The function call return {\em false} if the device could not be put into IP mode (this could be for example due to the device not supporting this drive mode at all).

\begin{verbatim}
canopen::sendPos(deviceID, pos);
canopen::sendSync(10);
\end{verbatim}
When a device is in IP mode, it can be moved by alternating between sending \texttt{canopen::sendPos} and \texttt{canopen::sendSync} commands. If a master thread is used, the user does not have to take care of sending SYNC commands.

\begin{verbatim}
canopen::enableBreak(deviceID);
canopen::shutdownDevice(deviceID);
canopen::closeConnection(); 
\end{verbatim}
These commands are used to put the motor state machine from operation into ready\_to\_switch\_on (i.e. enable the motor brake), to shut down the communication (301) state machine, and to close the device connection, respectively.

\subsection{Getting started: Communicating with multiple devices simultaneously (demo/multiple\_devices.cpp)}

This example shows how to communicate with a group (``chain'') of devices, e.g. a robot arm such as the {\em Schunk Powerball} arm. If you invoke the script with the CAN IDs (in decimal representation) of your devices as a command-line parameter, e.g. \texttt{./multiple\_devices 3 4 5 6 7 8}, you should first see all devices of the arm referencing themselves. Afterwards, you should see the arm performing some random motion.

If you compare \texttt{demo/single\_device} and \texttt{demo/multiple\_devices}, you will see the use of the API commands is completely identical in both cases. Of course, the API commands can also be used in this way to communicate simultaneously, for example, with an arm and a mobile base.

\section{CANopen with a master thread}

The {\em master thread} implemented in \texttt{canopenmaster.cpp} adds on top of the functionality seen in the previous examples. As has been seen above, its use is optional and the user is free to rely only on the message sending and parsing functionality implemented in the API.

The pre-implemented master thread can also be easily adapted or extended to accomodate other use cases (cf. Chapter \ref{chap:extending}).

The pre-implemented master also provides callback functions which are completely framework(e.g.ROS)-independent. These callbacks are intended to be wrapped easily for inter-process communication, as shown in the  {\em ROS\_canopen} package. 

In the framework-independent examples of this section, we simply call the 

\subsection{canopenmaster\_single\_device example}

In the file \texttt{demo/canopenmaster\_single\_device}, we show a very simple example of how to use the pre-implemented master thread. To show its use in a framework-independent way, we invoke the callbacks from a separate thread in the examples. 

\begin{verbatim}
#include <canopenmaster.h>
 \end{verbatim}
This is where the master thread functionality comes from.

\begin{verbatim}
 canopen::using_master_thread = true;
 \end{verbatim}
When using a master, you must set this namespace-global variable to {\em true}. Then any commands to send CAN messages are not sent directly to the bus, but rather are put on a sending queue which is controlled by the master. The master also has objects for all CAN devices and groups (chains) of CAN devices which have to be coordinated, e.g. a robot arm. The master will also take care of incoming PDOs from the devices and dispatch them to the device objects where they are used to update device object member variables (e.g. most recent position etc.) which can be queried at any time.

\begin{verbatim}
 canopen::initChainMap("/home/tys/git/other/canopen/demo/single_device.csv");
\end{verbatim}
A crucial part of the master is that it has an object for every device and for every group of devices (chain). The total of devices and chains is described in a small text file with one row per group (chain) of devices. Our minimalistic robot in this example consists only of a single device within a single chain. The config file partsed with the command above looks like this:

\begin{verbatim}
chain1 joint1 /dev/pcan32 12
\end{verbatim}
First entry in the row is the chain name. All other entries are triples (devicename devicebus deviceCANid). The command \texttt{canopen::initChainMap} takes this specification and creates a map (hash table), called \texttt{canopen::chainMap} that links chain names to (pointers to) ojects of class \texttt{canopen::Chain} (\texttt{chain.h}).

\begin{verbatim}
  if (!canopen::openConnection("/dev/pcan32")) {
    std::cout << "Cannot open CAN device; aborting." << std::endl;
    return -1;
  } 
  canopen::initNMT();
  canopen::initListenerThread();
\end{verbatim}
These commands are already known from the direct API call examples above and are used in exactly the same way here.

\begin{verbatim}
canopen::initIncomingPDOProcessorThread();
\end{verbatim}
Remember that without a master, incoming PDOs are simply stored in the queue \texttt{incomingPDOs}. Using a {\em master} thread implies having explicit object representations of devices and device groups (chains). These objects are created using the \texttt{canopen::initChainMap} command mentioned above. The command \texttt{initIncomingPDOProcessorThread} launches a thread that processes messages from this queue: based on the PDO cobID the PDOProcessor triggers a member function \texttt{canopen::Chain::updateStatusWithIncomingPDO} in the {\em canopen::Chain} object to which the device from which the PDO was sent belongs. This function can check if any object (index/subindex) from the object dictionary is contained in that PDO which it can use to update some of its status member variables. It then triggers a member function \texttt{canopen::Device::updateStatusWithIncomingPDO} of the object representing the device from which the PDO was sent. Again, this function can check for the presence of specific objects in the PDO using the member function \texttt{Message::find\_component} ({\bf todo}!) and update its status member variables. Currently, the only check performed is for the presence of the object \texttt{position\_actual\_value} (index 0x6064; mapped in the Schunk default tPDO), which if present in a PDO is used to update the member variable \texttt{current\_position\_}. Other checks and updates can be performed similarly and incorporated into the function \texttt{Device::updateStatusWithIncomingPDO} to extend the functionality.

\begin{verbatim}
canopen::initMasterThread();
\end{verbatim}




\begin{verbatim}
  while (true)
    std::this_thread::sleep_for(std::chrono::milliseconds(10));
\end{verbatim}

    

 




